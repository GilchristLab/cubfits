
\section[Main Functions]{Main Functions}
\label{sec:main_functions}
\addcontentsline{toc}{section}{\thesection. Main Functions}

The \pkg{cubfits} package is composed of three main functions:
\begin{enumerate}
\item \code{cubfits()} fits models for sequences with observed $\phi$ 
      (expression level) values and measurement errors,
\item \code{cubappr()} approximates models for sequences without any observed
      $\phi$ values, and
\item \code{cubpred()} fits models for sequences with observed $\phi$ values
      and measurement errors, and then predicts $\phi$ values for sequences
      without observations.
\end{enumerate}
See package's help pages for details of other input options.\footnote{
\code{?cubfits::cubfits}, \code{?cubfits::cubappr}, and
\code{?cubfits::cubpred}.
}


\subsection[Demonstrations]{Demonstrations}
\label{sec:demonstrtions}
\addcontentsline{toc}{subsection}{\thesubsection. Demonstrations}

The \pkg{cubfits} package provides quick examples for its three main functions:
\begin{Code}
> demo(roc.train, 'cubfits')    # for cubfits()
> demo(roc.appr, 'cubfits')     # for cubappr()
> demo(roc.pred, 'cubfits')     # for cubpred()
\end{Code}
These \pkg{cubfits} demos perform short MCMC runs
and analyze toy datasets (\code{ex.train} and \code{ex.test}) for the
Ribosome Overhead Cost (ROC) model~\citep{Shah2011}
which is shown in Figure~\ref{fig:plotbin}. The toy datasets have only 100
short sequences, and only 3 amino acids are considered.

For a standard data analysis, the process essentially consists of:
\begin{enumerate}
\item reading sequences and expressions files,
\item converting to appropriate data structures,
\item running a main function (MCMC),
\item summarizing MCMC outputs, and
\item plotting predictions.
\end{enumerate}
The \pkg{cubfits} package also provides an example using simulated data:
\begin{Code}
> demo(simu.roc, 'cubfits')     # cubfits() is called.
\end{Code}
Note that this demo will generate a fake sequence file (\code{toy_roc.fasta})
in FASTA format in the working directory, and (for testing)
read it back. Also, it converts the data into the correct format
needed by \code{cubfits()}, and finally runs MCMC and generates a plot.


\subsection[Generic Functions (Aside)]{Generic Functions (Aside)}
\label{sec:generic_functions}
\addcontentsline{toc}{subsection}{\thesubsection. Generic Functions (Aside)}

Note that the three main functions are wrappers of other generic functions that
perform parameter initializations, propose new parameters, compute
MCMC acceptance/rejection ratio, and more.
The function \code{init.function()} initializes generic functions that
will be called by the three main functions.
Although \code{init.function()} is called within each of three main
functions to setup the generic functions, it also needs to be called before
using other utility functions, such as \code{fitMultinom()}, see
Section~\ref{sec:misc} for examples.

Accompanying control variables such as \code{.CF.CT} and \code{.CF.OP},
the \code{init.function()} will dispatch the
corresponding generic functions into a default environment \code{.cubfitsEnv},
allowing other main functions may call those generic functions dynamically.

Note that generic functions in \proglang{R} typically only depend on
input object types.  However, the design in \pkg{cubfits} has 
several advantages:
\begin{itemize}
\item functions are clearer by making good use of data structures and
      simplifying options,
\item extensions are easier to create; simply add more generic functions 
      rather than changing main functions, and
\item performance is more efficient by avoiding extra
      conditional checks such as \code{if(...)\{...\} else\{...\}}
      in every iteration.
\end{itemize}
Also, the design can avoid tedious CRAN checks since there are some restrictions
in accessing \code{.GlobalEnv}. For example, 
\code{.cubfitsEnv$my.fitMultinomAll()} is called in several
internal functions to fit multinomial logistic regression in every MCMC
iterations. In particular, it has four generic functions:
\begin{enumerate}
\item \code{my.fitMultinomAll.lapply()} ueses \code{lapply()} in the serial version,
\item \code{my.fitMultinomAll.mclapply()} uses \code{parallel::mclapply()}
      in shared memory machines,
\item \code{my.fitMultinomAll.task.pull()} uses \code{pbdMPI::task.pull()}
      in distributed clusters, and
\item \code{my.fitMultinomAll.pbdLapply()} uses \code{pbdMPI::pbdLapply()}
      in distributed clusters, but is only efficient for homogeneous tasks.
\end{enumerate}
Through \code{init.function()}, there is no need to check which generic
function should be called within the MCMC step, and there is no need to worry
about serial or parallel details when designing a MCMC algorithm.

