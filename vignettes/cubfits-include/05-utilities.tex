
\section[Utilities]{Utilities}
\label{sec:utilities}
\addcontentsline{toc}{section}{\thesection. Utilities}

The \pkg{cubfits} package provides simplified data input and output utilities 
that can quickly import sequence and expression values, turn them into
\proglang{R} data structures, and dump analysis results in a standard format.
As long as data are in this common format, the following functions simplify
analysis work flows as in Section~\ref{sec:work_flows}.
Further, the \proglang{R} data structures used in \pkg{cubfits} can be
converted to different formats after reading in from disk and/or
before writing out to disk.


\subsection[Data I/O Functions]{Data I/O Functions}
\label{sec:data_io}
\addcontentsline{toc}{subsection}{\thesubsection. Data I/O Functions}

\begin{itemize}
\item \code{read.seq()} reads sequence data in FASTA format, and
\item \code{write.seq()} writes sequence data in FASTA format.
\end{itemize}
Both functions are simplified wrapper functions of package \pkg{seqinr}.
The writing function is mainly intended for simulation studies, as described in 
the section above.

\begin{itemize}
\item \code{read.phi()} reads in expression values in tsv/csv format, and
\item \code{write.phi()} writes out expression values in tsv/csv format.
\end{itemize}
The default data structure for these is an \proglang{R} \code{data.frame}.
The writing function is mainly for simulation studies.


\subsection[Main Data Structures]{Main Data Structures}
\label{sec:main_data_structures}
\addcontentsline{toc}{subsection}{\thesubsection. Main Data Structures}

The main data structures used in \pkg{cubfits} are:
\begin{itemize}
\item \code{reu13.df} (used by REU13 students)
      contains codon positions and expression levels,
\item \code{y} 
      (used by REU12 students and \citet{Wallace2013})
      contains synonymous codon counts for each sequence,
\item \code{n} 
      (used by REU12 students and \citet{Wallace2013})
      contains total codon count for each sequence,
\item \code{reu13.list} is a list version of \code{reu13.df} (adopted by WCC),
\item \code{y.list} is a list version of \code{y} (adopted by WCC), and
\item \code{n.list} is a list version of \code{n} (adopted by WCC).
\end{itemize}
These are typically fixed after data input steps and
mainly used in the three main functions without further changes.
Note that the objects with these data structures are normally sorted
by ORF id's or names. The list versions are also useful for some models
and speedup model fitting. Some conversion functions between data structures
are provided in \pkg{cubfits}, see
Section~\ref{sec:conversion_of_data_strcutures} for details.

Also, there are other data structures for parameters, but those are
sometimes model dependent. For example, data structure \code{b} may contain
model parameters in different dimension.
The parameters of each amino acid are
$(log(\mu), \Delta t)$ for ROC model,
$(log(\mu), \omega)$ for NSEf model, and
$(log(\mu), \Delta t, \omega)$ for ROC+NSEf model.
See help pages (\code{?cubfits::AllDataFormats}) for details.

The following is an example of \code{reu13.df} data structure containing
three amino acids A, C, and D. Each amino acid is in \code{data.frame}
with four required columns \code{ORF} (sequence id), \code{phi} (expression
level), \code{Pos} (codon position),
and \code{Codon}, and one optional column \code{Codon.id} (created by
\code{rearrange.reu13.df()} within \code{gen.reu13.df()} call.)
\begin{Code}[title=Example of \code{reu13.df}]
> str(ex.train$reu13.df)
List of 3
 $ A:'data.frame':	2682 obs. of  5 variables:
  ..$ ORF     : chr [1:2682] "YBL023C" "YBL023C" "YBL023C" "YBL023C" ...
  ..$ phi     : num [1:2682] 0.0186 0.0186 0.0186 0.0186 0.0186 ...
  ..$ Pos     : num [1:2682] 109 353 123 294 1133 ...
  ..$ Codon   : chr [1:2682] "GCA" "GCA" "GCA" "GCA" ...
  ..$ Codon.id: int [1:2682] 0 0 0 0 0 0 0 0 0 0 ...
 $ C:'data.frame':	662 obs. of  5 variables:
  ..$ ORF     : chr [1:662] "YBL023C" "YBL023C" "YBL023C" "YBL023C" ...
  ..$ phi     : num [1:662] 0.0186 0.0186 0.0186 0.0186 0.0186 ...
  ..$ Pos     : num [1:662] 387 862 813 248 226 40 82 477 922 87 ...
  ..$ Codon   : chr [1:662] "TGC" "TGC" "TGC" "TGT" ...
  ..$ Codon.id: int [1:662] 0 0 0 1 1 1 1 1 1 1 ...
 $ D:'data.frame':	3164 obs. of  5 variables:
  ..$ ORF     : chr [1:3164] "YBL023C" "YBL023C" "YBL023C" "YBL023C" ...
  ..$ phi     : num [1:3164] 0.0186 0.0186 0.0186 0.0186 0.0186 ...
  ..$ Pos     : num [1:3164] 209 199 255 89 273 141 263 158 112 306 ...
  ..$ Codon   : chr [1:3164] "GAC" "GAC" "GAC" "GAC" ...
  ..$ Codon.id: int [1:3164] 0 0 0 0 0 0 0 0 0 0 ...
\end{Code}

The following is an example of corresponding \code{y} which has only
synonymous codon counts for each amino acids and sequences.
\begin{Code}[title=Example of \code{y}]
> str(ex.train$y)
List of 3
 $ A: int [1:100, 1:4] 16 4 5 17 5 7 4 0 20 13 ...
  ..- attr(*, "dimnames")=List of 2
  .. ..$ : chr [1:100] "YBL023C" "YBL074C" "YBR060C" "YBR068C" ...
  .. ..$ : chr [1:4] "GCA" "GCC" "GCG" "GCT"
 $ C: int [1:100, 1:2] 3 0 1 5 4 2 4 2 0 11 ...
  ..- attr(*, "dimnames")=List of 2
  .. ..$ : chr [1:100] "YBL023C" "YBL074C" "YBR060C" "YBR068C" ...
  .. ..$ : chr [1:2] "TGC" "TGT"
 $ D: int [1:100, 1:2] 15 10 10 11 6 12 8 2 17 27 ...
  ..- attr(*, "dimnames")=List of 2
  .. ..$ : chr [1:100] "YBL023C" "YBL074C" "YBR060C" "YBR068C" ...
  .. ..$ : chr [1:2] "GAC" "GAT"
\end{Code}

The following is an example of corresponding \code{n} which has only
total codon counts for each amino acids and sequences.
\begin{Code}[title=Example of \code{n}]
> str(ex.test$n)
List of 3
 $ A: Named int [1:100] 60 22 19 16 22 28 9 22 35 8 ...
  ..- attr(*, "names")= chr [1:100] "YAL017W" "YBL033C" "YBL102W" "YBR010W" ...
 $ C: Named int [1:100] 12 7 5 0 1 7 5 4 4 4 ...
  ..- attr(*, "names")= chr [1:100] "YAL017W" "YBL033C" "YBL102W" "YBR010W" ...
 $ D: Named int [1:100] 96 26 6 4 24 21 5 28 28 3 ...
  ..- attr(*, "names")= chr [1:100] "YAL017W" "YBL033C" "YBL102W" "YBR010W" ...
\end{Code}


\subsection[Conversion of Data Structures]{Conversion of Data Structures}
\label{sec:conversion_of_data_strcutures}
\addcontentsline{toc}{subsection}{\thesubsection. Conversion of Data Structures}

In order to improve performance, simply parallelizing the code is not enough.
In \proglang{R}, a good data structure can have massive impact on performance.
In practice, it is best that not to change/modify/subset existing
data structures within computations/iterations.
% As long as data are not too big and in manageable size,
% for example in summarized statistics as Section~\ref{sec:main_data_structures}.
It can be worth generating extra data structures which are efficient
for some specific functions or particular computation, even if the information
is at times redundant.

However, managing different data structures could be a nightmare in maintaining
consistency across function calls.
\pkg{cubfits} provides several utilities to generate (\code{gen*()})
or convert (\code{convert*()}) extended data structures, such as
\code{reu13.list}, \code{n.list}, and \code{y.list}.
See help pages (\code{?cubfits::DataGenerating} and
\code{?cubfits::DataConverting}) for details.

Finally, \pkg{cubfits} provides a simple demo
\begin{Code}
> demo(basic, 'cubfits')
\end{Code}
and shows how those generating and converting functions work.
Note that this demo reads two example files from \pkg{cubfits} and turns them
into main data structures which can be called by the three main functions 
introduced in Section~\ref{sec:main_functions}.
The files are \code{seq_200.fasta} containing 200 sequences and
\code{phi_200.tsv} containing the corresponding expression levels.
Both files are stored in the package directory \code{cubfits/inst/ex_data/}
and installed in \code{$\{R_HOME\}/library/cubfits/ex_data/}.

