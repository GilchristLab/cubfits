
\section[FAQ]{FAQ}
\label{sec:faq}
\addcontentsline{toc}{section}{\thesection. FAQ}

\begin{enumerate}

\item {\bf\color{blue} Q:}
      Should the amino acids and ORFs be sorted for \pkg{cubfits}? \\
      {\bf\color{blue} A:}
      Yes. For performance issue, amino acids are coded with one character
      code, and ORFs and their expression data (if any) should be named.
      Both should be sorted by their names.

\item {\bf\color{blue} Q:}
      What is the main difference of \code{cubfits()}, \code{cubappr()}, and
      \code{cubpred()}? \\
      {\bf\color{blue} A:}
      \code{cubfits()} is the usual MCMC method to estimate parameters where
      some people call backward simulation.
      On the other hand, \code{cubappr()} is pure simulation without
      observations from equilibrium states (if they have been reached.)
      \code{cubpred()} is more like machine learning techniques, such as
      cross-validation methods, use part of observations to estimate
      parameters, and predict the other parts of data.

\item {\bf\color{blue} Q:}
      Should \code{phi.Obs} be scaled? \\
      {\bf\color{blue} A:}
      For \code{cubappr()}, it must be scaled to mean 1 since the function uses
      as \code{phi.Obs} as initial values of $\phi_g$.
      For \code{cubfits()} and \code{cubpred()}, it is not necessary. The
      bias terms should be also estimated if \code{phi.Obs} were not
      scaled to mean 1. Post scaling of each MCMC iteration should be
      performed on parameters and prediction of $\phi_g$ to make them
      comparable with \code{cubappr()} results. However, this may induce
      bias as well.

\item {\bf\color{blue} Q:}
      What are configurations to enforce estimating bias of $\phi_g$,
      $K_{bias}$? \\
      {\bf\color{blue} A:}
      The only options currently \pkg{cubfits} have are in next.
\begin{Code}[title=Configuration]
  .CF.CT$type.p <- "lognormal_bias"
  .CF.CT$scale.phi.Obs <- FALSE
  .CF.CONF$estimate.bias.Phi <- TRUE
\end{Code}

\item {\bf\color{blue} Q:}
      What is the default environment of \pkg{cubfits}? \\
      {\bf\color{blue} A:}
      \code{.cubfitsEnv} will dynamicly store generic functions, while
      all data are still located in \code{.GlobalEnv}.
      See Section~\ref{sec:misc} for examples.

\item {\bf\color{blue} Q:}
      What are the ways to access \pkg{cubfits} functions? \\
      {\bf\color{blue} A:}
      There are at least three levels of functions developed in \pkg{cubfits}:
      \begin{itemize}
        \item \code{cubfits::function\_name()} or simply \code{function\_name()}
              (if \pkg{cubfits} is loaded) can access exported major functions
              of \pkg{cubfits},
        \item \code{cubfits:::function\_name()}
              can access unexported internal functions and objects
              of \pkg{cubfits} (even \pkg{cubfits} is not loaded), and
        \item \code{.cubfitsEnv$function\_name()}
              can access generic functions dispatched in the environment
              \code{.cubfitsEnv} if \code{init\_function()} has been called.
              (\code{ls(.cubfitsEnv)} can see what have been dispatched.)
      \end{itemize}

\item {\bf\color{blue} Q:}
      What is the way to debug \pkg{cubfits} functions? \\
      {\bf\color{blue} A:}
      Debugger of negative \proglang{R} may not work well in some cases, so we
      suggest:
      \begin{itemize}
        \item \code{cat()} or \code{print()} and \code{source()}
              can be useful for non-generic functions, and
        \item \code{browser()} would be better for generic functions
              in \code{.cubfitsEnv}.
      \end{itemize}
      Note that if you \code{source()} a new code into \proglang{R}, you
      may still have to overwrite functions in \code{.cubfitEnv} since
      MCMC iterations heavily access functions from there instead of
      \code{.GlobalEnv}.

\end{enumerate}

